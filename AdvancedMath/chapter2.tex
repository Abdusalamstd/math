\mychapter{فۇنكسىيە ۋە لىمىت نەزەرىيىسى}
\section{فۇنكىسىيە}
فۇنكسىيەنىڭ ئېنىقلىمىسى ئادەتتە ئەنئەنىۋى ئېنىقلىما ۋە زامانىۋى ئېنىقلىما دەپ ئىككىگە بۆلىنىدۇ. باياننىڭ ئۇقۇمىنىڭ باشلىنىش نۇقتىسى باشقىچە بولغاندىن باشقا ، فۇنكىسىيەنىڭ ئىككى ئېنىقلىمىسى ئاساسەن ئوخشاش. ئەنئەنىۋى ئېنىقلىما ھەرىكەت ئۆزگىرىشى نۇقتىسىدىن باشلىنىدۇ. ، زامانىۋى ئېنىقلىما بولسا توپلام ۋە ئەكىس ئېتىش نۇقتىسىدىن باشلىنىدۇ.

\begin{MyDefinition}{فۇنكىسىيەنىڭ ئېنىقلىمىسى}{}
 خالىغان توپلام $A$ دىكى ئېلمىنىت $x$ گە نىسبەتەن، ماسلىق مۇناسىۋىتى $f$ مەۋجۇت بولۇپ، بۇ $x$ گە تەسىر قىلىغاندىن كىيىن ئېرىشكەن توپلام $B$ نىڭ ئېلمىنتى $y$ بولسا، ئۇنداقتا $f(x)$ بولسا توپلام $A$ دىن توپلام $B$ غا بولغان ئەكىس ئېتىش ھېساپلىنىدۇ. بۇنىڭدا $y$ بولسا $x$ نىڭ فۇنكىسىيەسى دېيىلىدۇ. بۇنى 
 $$x:\rightarrow y \Leftrightarrow y=f(x)$$
 ئارقىلىق خاتېرلەشكە بولىدۇ.
\end{MyDefinition}
فۇنكىسىيە ئۇقۇمى ئۈچ دائىرىنى ئۆز ئىچىگە ئالىدۇ: ئېنىقلىما ساھەسى $A$ ، قىممەت دائىرىسى $B$ ۋە مۇناسىۋەت ئىپادىسى $f$

\section{سانلار قاتارى}
\subsection{تەڭ ئايرىمىلىق سانلار ئارقىمۇ-ئارقىلىقى}
\subsection{تەڭ نىسبەتلىك سانلار ئارقىمۇ-ئارقىلىقى}
\subsection{سانلار قاتارى}

\section{لىمىت}
\subsection{لىمىت ۋە ئۇنىڭ خۇسۇسىيەتلىرى}
\subsection{سانلار قاتارى لىمىتى}
\subsection{فۇنكىسىيە لىمىتى}
\subsection{سانلار قاتارى ۋە فۇنكىسىيە لىمىتى}

\section{فۇنكىسىيە ئۈزلۈكسىزلىكى}
\subsection{فۇنكىسىيە مونوتونلىقى}
\subsection{فۇنكىسىيە ئۈزۈك نۇقتىسى}
