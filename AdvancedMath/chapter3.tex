\mychapter{دىففېرېنسىيال ۋە ئىنتېگرال}
\par\bigskip
\begin{tcolorbox}
	\textcolor{blue}{\textbf{بۇ باپتىكى مۇھېم نۇقتىلار:}}
دىففېرېنسىيال ۋە ئىنتېگرال ماتىماتىكا پېنىدىكى ئەڭ يادرولۇق بىلىم نۇقتىسىنىڭ بىرى، شۇنداقلا ئۇ فىزىكا ئىلمىنىڭ ئاساسى. بۇ باپتا ھاسىلە ئۇقۇمىدىن باشلاپ دىففېرېنسىيال ۋە ئىنتېگرالنىڭ ئەمەلىي قوللىنىشىغىچە بولغان مەزمۇنلار خاتىرلىنىدۇ. بۇ باپتا يەنە دىففېرېنسىيال ۋە ئىنتېگرالغا ئائىت ھېسابلاشلار، تېئورېمىلار چۈشەندۈرىلىدۇ. بۇنىڭدىن سىرت دىففېرېنسىيال ۋە ئىنتېگرالنىڭ ماھىيىتى ۋە قوللىنىلىشىغا دائىر ئەمەلىي مىساللار تونۇشتۇرلىدۇ.
\end{tcolorbox}

\section{ھاسىلە ئۇقۇمى}
تىزلىك ئۇقۇمى ھەممە كىشىگە ئەڭ تونۇش بولسا كىرەك، مەسىلەن ماشىنا تىزلىكى $60 km/h$ دىگەندەك. بۇ يەردىكى  $60 km/h$ ئەمەلىيەتتە  $16.67 m/s$ بىلەن ئوخشاش. يەنى 1 سېكۇنتتا 16.67 مىتېر يۆتكىلىدۇ دىگەنلىك. تىزلىك قانداق قىلىپ ماڭغان يۆتكىلىشكە ئايلاندى؟ ماڭغان يۆتكىلىش قانداق قىلىپ تېزلىكنى مەۋجۇت قىلدى؟ جىسىم مەلۇم تېزلىككە يىتىش ئۈچۈن تىنىچ ھالەتتىن قوزغالسا، ئۇ نىشان تېزلىككە يەتكۈچە قانداق قىلىپ تىزلىنىشنى مەۋجۇت قىلدى؟ بۇنداق فىزىكىلىق ھادىسىلەرنى تەتقىق قىلىشتا ھاسىلىگە تايانماي بولمايدۇ.

فۇنكىسىيە 
$f(x)$
دە، ئىككى نۇقتا
$x_1,x_2$
دىن ئۆتكەن سىزىقنىڭ يانتۇلۇقى (يانتۇلۇق بۇلۇڭى $\alpha$ نىڭ تانگېنىس قىممىتى)
نى مۇنداق ئىپادىلەشكە بولىدۇ:
$$\tan\alpha = \frac{f(x_2)-f(x_1)}{x_2-x_1}$$

 بىر نۇقتا
$x_0$
 نىڭ قوشنا دائىرىسى  ئىچىدە، نۇقتا
$x_0$
دىكى ئۇرۇنما سىزىقىنىڭ يانتۇلۇقىنى مۇنداق ئىپادىلەشكە بولىدۇ:
$$\tan\alpha = \lim\limits_{x\to x_0}\frac{f(x)-f(x_0)}{x-x_0}$$
گەرچە 
$\lim\limits_{x\to x_0}{x-x_0} =0$
بولسىمۇ،
$\lim\limits_{x\to x_0}{(f(x)-f(x_0))}$
مۇ نۆلگە چەكسىز يېقىنلىشىدۇ، شۇڭا بۇلارنىڭ نىسبەتلىرى 0 بولۇپ كېتىشى ناتايىن.

\begin{MyDefinition}{ھاسىلە}{}
فۇنكىسىيە $f(x)$ ئېنتېرۋال $I$ دا، ئۆزگەرگۈچى مىقدار $x$ گە بولغان ئارتقۇچى مىقدار
$\Delta x$
گە نىسبەتەن، ئېنتېرۋال $I$ دىكى خالىغان بىر نۇقتا 
$x=x_0$ 
ئۈچۈن،
$x_0 \in I, x_0+\Delta x \in I$
شەرتىنى قانائەتلەندۈرسە،فۇنكىسىيەنىڭ ئارتقۇچى مىقدارى 
$\Delta y = f(x_0+\Delta x)-f(x_0)$
بولىدۇ،ئەگەر ئارتقۇچى مىقدار 
$\Delta x \to 0$
بولغاندا، ئۇنىڭ
$\Delta y$
گە بولغان نىسبىتى مەۋجۇت بولسا، بۇ نىسبەتنى بىز  $f(x)$ نىڭ 
$x_0$
نۇقتىدىكى \textbf{ھاسىلە}سى دەپ ئاتايمىز، ھەم ئۇنى $f'(x_0)$ قىلىپ خاتىرلەيمىز. بۇنىڭدا
$$
f'(x_0) = \lim\limits_{\Delta x \to 0}\frac{\Delta y}{\Delta x}
= \lim\limits_{\Delta x \to 0}\frac{f(x_0+\Delta x)-f(x_0)}{\Delta x}
$$

\end{MyDefinition}
تۆۋەندىكى جۈملىلەر تەڭداش:
\begin{enumerate}
	\item[\faIcon{hand-point-left}] $y=f(x)$ نۇقتا $x_0$ دا ھاسىلىسى بار
	\item[\faIcon{hand-point-left}] $y=f(x)$ نۇقتا $x_0$ دا ھاسىلىسى مەۋجۇت
	\item[\faIcon{hand-point-left}] $f'(x)=A$ ($A$ چەكلىك)
\end{enumerate}

\begin{indicate}
	\begin{minipage}[b]{0.85\linewidth}
يەككە ھاسىلىسى بولسا، ئوڭ ياكى سولدىكى ھاسىلىسىنى  كۆرسىتىدۇ، مەسىلەن:
$$f'_+(x)=\lim\limits_{\Delta x\to 0^+}\dfrac{\Delta y}{\Delta x}=\lim\limits_{\Delta x\to 0}\dfrac{f(x_0+\Delta x)-f(x_0)}{\Delta x}$$
$$f'_-(x)=\lim\limits_{\Delta x\to 0^-}\dfrac{\Delta y}{\Delta x}=\lim\limits_{\Delta x\to 0}\dfrac{f(x_0+\Delta x)-f(x_0)}{\Delta x}$$
	
	\end{minipage}
	\hfil
	\begin{minipage}[b]{0.1\linewidth}
		\begin{tikzpicture}
			\node[graduate,minimum size=1.5cm]{};
		\end{tikzpicture}
	\end{minipage}
\end{indicate}

\subsection{فۇنكىسىيە ھاسىلىسى}
\subsection{يۇقرى دەرىجىلىك ھاسىلە}

\section{دىففېرېنسىيال}
\begin{figure}[htp]
  \centering
  \begin{tikzpicture}
    \coordinate (p1) at (0.7,3); 
    \coordinate (p2) at (1,3.3);
    \coordinate (p3) at (2,2.5); 
    \coordinate (p4) at (3,2.5);
    \coordinate (p5) at (4,3.5); 
    \coordinate (p6) at (5,4.1);
    \coordinate (p7) at (6,3.4); 
    \coordinate (p8) at (7,4.1);
    \coordinate (p9) at (8,4.6); 
    \coordinate (p10) at (9,4);
    \coordinate (p11) at (9.5,4.7);

    % The cyan background
    \fill[cyan!10] (p2|-0,0) -- (p2) -- (p3) -- (p4) -- (p5) -- (p6)
    -- (p7) -- (p8) -- (p9) -- (p10) -- (p10|-0,0) -- cycle;
    % the dark cyan stripe
    \fill[cyan!30] (p6|-0,0) -- (p6) -- (p7) -- (p7|-0,0) -- cycle;
    % the curve
    \draw[thick,cyan] (p1) to[out=70,in=180] (p2) to[out=0,in=150]
    (p3) to[out=-50,in=230] (p4) to[out=30,in=220] (p5)
    to[out=50,in=150] (p6) to[out=-30,in=180] (p7) to[out=0,in=230]
    (p8) to[out=40,in=180] (p9) to[out=-30,in=180] (p10)
    to[out=0,in=260] (p11);
    % the broken line connecting points on the curve
    \draw (p2) -- (p3) -- (p4) -- (p5) -- (p6) -- (p7) -- (p8) -- (p9)
    -- (p10);

    % the h line
    \draw[red,thick] ($(p6) + (0.0, -3.0)$) --
    node[above,red]{$h$}($(p6) + (1.0, -3.0)$);
    \node[above,yshift=0.5cm,text=black] at (p6) {$h=\frac{b-a}{n}$};
        
    % vertical lines and labels
    \foreach \n/\texto in { 2/{a=x_0}, 3/{x_1}, 4/{}, 5/{},
      6/{x_{j-1}}, 7/{x_j}, 8/{}, 9/{x_{n-1}}, 10/{b=x_n} } { \draw
      (p\n|-0,0) -- (p\n); \node[below,text height=1.5ex,text
      depth=1ex,font=\small] at (p\n|-0,0) {$\texto$}; }
        
    % The axes
    \draw[->] (-0.5,0) -- (10,0) coordinate (x axis); \draw[->]
    (0,-0.5) -- (0,6) coordinate (y axis);
    % labels for the axes
    \node[below] at (x axis) {$x$}; \node[left] at (y axis) {$y$};
    % label for the function
    \node[above,text=cyan] at (p11) {$y=f(x)$};

    % label for the f(a)和f(b)
    \node[above,text=black] at (p2) {$f(a)$}; \node[above,text=black]
    at (p10) {$f(b)$};
  \end{tikzpicture}
  \caption{دېففېرىرېنسىئال}\label{fig-integrate}
\end{figure}
\subsection{فۇنكىسىيە دىففېرېنسىيالى}
\subsection{ھاسىلە فورمۇلىسى}

\textbf{دەرىجىلىك فۇنكىسىيە}

%\begin{tabular}{m{3cm}<{\centering}|m{7cm}<{\centering}|m{5cm}<{\centering}}
\begin{table}[H]
	\centering
	%\rowcolors{2}{cyan!35}{}
	\begin{tabular}{cccc}
		%\rowcolor{black!20}
		\toprule
		ئەسلىسى & ھاسىلىسى & ئەسلىسى & ھاسىلىسى \\ 
		\midrule
		$C$ & $0$ & $n^x$ & $n^x\ln n$ \\\hline
	$\log_ax$ & $\dfrac{1}{x\ln a}$ & $\ln x=\ln\vert x\vert$ & $\dfrac{1}{x}$ \\\hline
		$x^n$ & $nx^{n-1}$ & $\sqrt[n]{x}$ & $\dfrac{x^{-\frac{n-1}{n}}}{n}$ \\\hline
		$\dfrac{1}{x^n}$ & $-\dfrac{n}{x^{n+1}}$ & & \\
		\bottomrule
\end{tabular}
\caption{دەرىجىلىك فۇنكىسىيە ھاسىلىسى}
\end{table}


\textbf{تىرگىنومېترىيەلىك فۇنكىسىيە}
\begin{table}[H]
	\centering
	%\rowcolors{2}{cyan!35}{}
	\begin{tabular}{cccc}
		%\rowcolor{black!20}
		\toprule
		ئەسلىسى & ھاسىلىسى & ئەسلىسى & ھاسىلىسى \\ 
		\midrule
        $\sin x$ & $\cos x$ & $\cos x$ & $-\sin x$ \\ \hline
$\tan x$ & $\dfrac{1}{\cos^2x}=\sec^2x$ & $\cot x$ & $\dfrac{1}{\sin^2x}=\csc^2x$ \\ \hline
$\sec x$ & $\sec x\tan x$ & $\csc x$ & $-\csc x\cot x$ \\ \hline
$\arcsin x$ & $\dfrac{1}{\sqrt{1-x^2}}$ & $\arccos x$ & $-\dfrac{1}{\sqrt{1-x^2}}$ \\ \hline
$\arctan x$ & $\dfrac{1}{1+x^2}$ & $\textrm{arccot}\,x$ & $-\dfrac{1}{1+x^2}$ \\ \hline
$\textrm{arcsec}\,x$ & $\dfrac{1}{x\sqrt{x^2-1}}$ & $\textrm{arccsc}\,x$ & $-\dfrac{1}{x\sqrt{x^2-1}}$ \\\hline
$\textrm{sinh}\,x$ & $\textrm{cosh}\,x$ & $\textrm{cosh}\,x$ & $\textrm{sinh}\,x$ \\ \hline
$\textrm{tanh}\,x$ & $\dfrac{1}{\textrm{cosh}\,x^2}$ & $\textrm{arcsinh}\,x$ & $\dfrac{1}{\sqrt{x^2+1}}$ \\ \hline
$\textrm{arccosh}\,x$ & $\dfrac{1}{\sqrt{x^2-1}}$ & $\textrm{arctan}\,x$ & $\dfrac{1}{1-x^2}$ \\
		\bottomrule
	\end{tabular}
	\caption{تىرگىنومېترىيەلىك فۇنكىسىيە ھاسىلىسى}
\end{table}


\section{دىففېرېنسىيال تېئورمىسى}
\subsection{فېرمات تېئورمىسى}
\subsection{لور تېئورمىسى}
\subsection{لاگرانج تېئورمىسى}
\subsection{كوشى تېئرمىسى}
\subsection{دىففېرېنسىيال ئوتتۇرا قىممەت تېئورمىسى}

\section{تەيلېر يېيىلمىسى}
\subsection{تەيلېر يىيېلمىسى}
\subsection{تەيلېر فورمۇلىسى}

\section{فۇنكىسىيە خۇسۇسىيىتى}
\subsection{فۇنكىسىيە يىلتېزى}
\subsection{فۇنكىسىيە مونوتون رايونى}
\subsection{فۇنكىسىيە ئېكىستېرمۇم قىممىتى}
\subsection{فۇنكىسىيە كۆپۈنگۈ ۋە پېتىنقى قىسمى}
\subsection{فۇنكىسىيە بۇرۇلۇش نۇقتىسى}

\section{ياي دىففېرېنسىيالى}
\subsection{ياي دىففېرېنسىيالى}
\subsection{ئەگرىلىك}
\subsection{ئەگرىلىك رادېئۇس}
