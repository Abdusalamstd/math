\mychapter{دىففېرېنسىيال ۋە ئىنتېگرال}
\par\bigskip
\begin{tcolorbox}
	\textcolor{blue}{\textbf{بۇ باپتىكى مۇھېم نۇقتىلار:}}
دىففېرېنسىيال ۋە ئىنتېگرال ماتىماتىكا پېنىدىكى ئەڭ يادرولۇق بىلىم نۇقتىسىنىڭ بىرى، شۇنداقلا ئۇ فىزىكا ئىلمىنىڭ ئاساسى. بۇ باپتا ھاسىلە ئۇقۇمىدىن باشلاپ دىففېرېنسىيال ۋە ئىنتېگرالنىڭ ئەمەلىي قوللىنىشىغىچە بولغان مەزمۇنلار خاتىرلىنىدۇ. بۇ باپتا يەنە دىففېرېنسىيال ۋە ئىنتېگرالغا ئائىت ھېسابلاشلار، تېئورېمىلار چۈشەندۈرىلىدۇ. بۇنىڭدىن سىرت دىففېرېنسىيال ۋە ئىنتېگرالنىڭ ماھىيىتى ۋە قوللىنىلىشىغا دائىر ئەمەلىي مىساللار تونۇشتۇرلىدۇ.
\end{tcolorbox}

\section{ھاسىلە ئۇقۇمى}
تىزلىك ئۇقۇمى ھەممە كىشىگە ئەڭ تونۇش بولسا كىرەك، مەسىلەن ماشىنا تىزلىكى $60 km/h$ دىگەندەك. بۇ يەردىكى  $60 km/h$ ئەمەلىيەتتە  $16.67 m/s$ بىلەن ئوخشاش. يەنى 1 سېكۇنىتتا 16.67 مىتېر يۆتكىلىدۇ دىگەنلىك. تىزلىك قانداق قىلىپ ماڭغان يۆتكىلىشكە ئايلاندى؟ ماڭغان يۆتكىلىش قانداق قىلىپ تىزلىكنى مەۋجۇت قىلدى؟ جىسىم مەلۇم تىزلىككە يىتىش ئۈچۈن تىنىچ ھالەتتىن قوزغالسا، ئۇ نىشان تىزلىككە يەتكۈچە قانداق قىلىپ تىزلىنىشنى مەۋجۇت قىلدى؟ بۇنداق فىزىكىلىق ھادىسىلەرنى تەتقىق قىلىشتا ھاسىلىگە تايانماي بولمايدۇ.
\subsection{فۇنكىسىيە ھاسىلىسى}
\subsection{يۇقچى دەرىجىلىك ھاسىلە}

\section{دىففېرېنسىيال}
\begin{figure}[htp]
  \centering
  \begin{tikzpicture}
    \coordinate (p1) at (0.7,3); 
    \coordinate (p2) at (1,3.3);
    \coordinate (p3) at (2,2.5); 
    \coordinate (p4) at (3,2.5);
    \coordinate (p5) at (4,3.5); 
    \coordinate (p6) at (5,4.1);
    \coordinate (p7) at (6,3.4); 
    \coordinate (p8) at (7,4.1);
    \coordinate (p9) at (8,4.6); 
    \coordinate (p10) at (9,4);
    \coordinate (p11) at (9.5,4.7);

    % The cyan background
    \fill[cyan!10] (p2|-0,0) -- (p2) -- (p3) -- (p4) -- (p5) -- (p6)
    -- (p7) -- (p8) -- (p9) -- (p10) -- (p10|-0,0) -- cycle;
    % the dark cyan stripe
    \fill[cyan!30] (p6|-0,0) -- (p6) -- (p7) -- (p7|-0,0) -- cycle;
    % the curve
    \draw[thick,cyan] (p1) to[out=70,in=180] (p2) to[out=0,in=150]
    (p3) to[out=-50,in=230] (p4) to[out=30,in=220] (p5)
    to[out=50,in=150] (p6) to[out=-30,in=180] (p7) to[out=0,in=230]
    (p8) to[out=40,in=180] (p9) to[out=-30,in=180] (p10)
    to[out=0,in=260] (p11);
    % the broken line connecting points on the curve
    \draw (p2) -- (p3) -- (p4) -- (p5) -- (p6) -- (p7) -- (p8) -- (p9)
    -- (p10);

    % the h line
    \draw[red,thick] ($(p6) + (0.0, -3.0)$) --
    node[above,red]{$h$}($(p6) + (1.0, -3.0)$);
    \node[above,yshift=0.5cm,text=black] at (p6) {$h=\frac{b-a}{n}$};
        
    % vertical lines and labels
    \foreach \n/\texto in { 2/{a=x_0}, 3/{x_1}, 4/{}, 5/{},
      6/{x_{j-1}}, 7/{x_j}, 8/{}, 9/{x_{n-1}}, 10/{b=x_n} } { \draw
      (p\n|-0,0) -- (p\n); \node[below,text height=1.5ex,text
      depth=1ex,font=\small] at (p\n|-0,0) {$\texto$}; }
        
    % The axes
    \draw[->] (-0.5,0) -- (10,0) coordinate (x axis); \draw[->]
    (0,-0.5) -- (0,6) coordinate (y axis);
    % labels for the axes
    \node[below] at (x axis) {$x$}; \node[left] at (y axis) {$y$};
    % label for the function
    \node[above,text=cyan] at (p11) {$y=f(x)$};

    % label for the f(a)和f(b)
    \node[above,text=black] at (p2) {$f(a)$}; \node[above,text=black]
    at (p10) {$f(b)$};
  \end{tikzpicture}
  \caption{دېففېرىرېنسىئال}\label{fig-integrate}
\end{figure}
\subsection{فۇنكىسىيە دىففېرېنسىيالى}
\subsection{ھاسىلە فورمۇلىسى}


\section{دىففېرېنسىيال تېئورمىسى}
\subsection{فېرمات تېئورمىسى}
\subsection{لور تېئورمىسى}
\subsection{لاگرانج تېئورمىسى}
\subsection{كوشى تېئرمىسى}
\subsection{دىففېرېنسىيال ئوتتۇرا قىممەت تېئورمىسى}

\section{تەيلېر يېيىلمىسى}
\subsection{تەيلېر يىيېلمىسى}
\subsection{تەيلېر فورمۇلىسى}

\section{فۇنكىسىيە خۇسۇسىيىتى}
\subsection{فۇنكىسىيە يىلتېزى}
\subsection{فۇنكىسىيە مونوتون رايونى}
\subsection{فۇنكىسىيە ئېكىستېرمۇم قىممىتى}
\subsection{فۇنكىسىيە كۆپۈنگۈ ۋە پېتىنقى قىسمى}
\subsection{فۇنكىسىيە بۇرۇلۇش نۇقتىسى}

\section{ياي دىففېرېنسىيالى}
\subsection{ياي دىففېرېنسىيالى}
\subsection{ئەگرىلىك}
\subsection{ئەگرىلىك رادېئۇس}
