\mychapter{دىففېرېنسىئال تەڭلىمە}
\par\bigskip
\begin{tcolorbox}
تەڭلىمە ئۇقۇمى باشلانغۇچ ماتېماتىكىسىدا ئەڭ بۇرۇن ئۇچرايدۇ. ئىلگىرىكى مەزمۇنلاردا فۇنكسىيە، ھاسىلە ئۇقۇمى، دىففېرېنسىئال ۋە ئىنتېگرال ئۇقۇملىرىنى ئىگلىگەندىن كىيىن مۇشۇلارنىڭمۇ تەڭلىمىگە ئائىت قوللىنىشلىرىنى بىلىش ئۈچۈن، شۇنداقلا تۇرمۇشتىكى ئەمەلىي مەسىلىلەرنىڭ ئېھتىياجى ئۈچۈن تۆۋەندە يېڭى بىر بىلىم نۇقتىسى بىلەن تونۇشۇپ چىقىمىز. بۇ باپتا بىر قەدەر قىيىن بولغان نۇقتا
\textbf{دىففېرېنسىئال تەڭلىمە}
ھەققىدە دەسلەپكى بىلىملەرنى ئۆگىنىپ چىقايلى.

\end{tcolorbox}
\section{دىففېرېنسىئال تەڭلىمە}
دىففېرېنسىيال تەڭلىمە نامەلۇم فۇنكسىيەنىڭ ھاسىلىسى بىلەن ئەركىن ئۆزگەرگۈچى مىقدار ئوتتۇرىسىدىكى مۇناسىۋەت سىستېمىسىنى تەسۋىرلەيدىغان تەڭلىمىنى كۆرسىتىدۇ. دىففېرېنسىئال تەڭلىمىنىڭ يېشىمى تەڭلىمىگە ماس كېلىدىغان فۇنكسىيە بولىدۇ. ھالبۇكى، ئبلبمبنتار ماتبماتىكىنىڭ ئالگببرالىق تەڭلىمىسىنىڭ يېشىمى تۇراقلىق سانلىق قىممەتتۇر. 
\subsection{ئاساسىي ئۇقۇم}

\begin{MyDefinition}{دىففېرېنسىئال تەڭلىمە}{}
	نامەلۇم فۇنكىسىيە ۋە نامەلۇم فۇنكىسىيە ھاسىلىسىنىڭ ئۆزگەرگۈچى مىقدار ئارىسىدىكى مۇناسىۋىتىنى ئىپادىلەيدىغان تەڭلىمە. يەنى فۇنكىسىيە ھاسىلىسىنى ئۆز ئىچىگە ئالغان تەڭلىمە دىففېرېنسىئال تەڭلىمە دەپ ئاتىلىدۇ. بۇنى $$F(x,y,y',...,y^{(n)})=0$$ ئارقىلىق خاتېرلەشكە بولىدۇ.
\end{MyDefinition}
دىففېرېنسىئال تەڭلىمىدىكى نامەلۇم فۇنكىسىيە ھاسىلىسىنىڭ دەرىجىسى، دىففېرېنسىئال تەڭلىمىنىڭ \textbf{دەرىجىسى} دەپ ئاتىلىدۇ.

\subsubsection{دىففېرېنسىئال تەڭلىمىنىڭ يېشىمى}
ئەگەر فۇنكىسىيە $y=\phi(x)$ نىڭ $n$ دەرىجىلىك ئۈزلۈكسىز ھاسىلىسى $\phi^n(x)$ ، بېرىلگەن ئىنتېرۋال $I$ دا مەۋجۇت ھەم تەڭلىمە
$$F(x,\phi(x),\phi'(x),...,\phi^{(n)}(x))=0$$
نى قانائەتلەندۈرسە، ئۇنداقتا فۇنكىسىيە $y=\phi(x)$ تەڭلىمە
$$F(x,\phi(x),\phi'(x),...,\phi^{(n)}(x))=0$$
نىڭ ئىنتېرۋال $I$ دىكى \textbf{يېشىمى} دەپ ئاتىلىدۇ. 
\subsubsection{ئومۇمىي يېشىمى}
ئەگەر دىففېرېنسىئال تەڭلىمە يېشىمى خالىغان تۇراقلىق ساننى ئۆز ئىچىگە ئالغان ھەمدە خالىغان تۇراقلىق ساننىڭ سانى تەڭلىمە دەرىجىسى بىلەن تەڭ بولغاندا، بۇ يېشىمىنى تەڭلىمىنىڭ \textbf{ئومۇمىي يېشىمى} دەپ ئاتايمىز.
\subsubsection{ئالاھېدە يېشىمى}
دىففېرېنسىئال تەڭلىمە ئومۇمىي يېشىمىدىكى خالىغان تۇراقلىق ساننى مۇقىم بېكىتكەندىن كىيىن ئېرىشكەن يېشىمنى، \textbf{ئالاھىدە يېشىمى} دەپ ئاتايمىز.

\subsection{ئاساسىي تەڭلىمىلەر}
\begin{itemize}
\item \textbf{دەسلەپكى قىممەت شەرتى}\\
ئەگەر $x=x_0$ بولغاندىكى فۇنكىسىيە ۋە ئۇنىڭ ھاسىلىسىنىڭ قىممىتى
 $y_0,y_{0}'$ بېرىلگەن بولسا، بۇنداق شەرتلەرنى بىز تەڭلىمىنىڭ دەسلەپكى قىممەت شەرتى دەپ ئاتايمىز.
\item \textbf{بىرىنچى دەرىجىلىك دەسلەپكى قىممەت مەسىلىسى}\\
تەڭلىمە $y'=f(x,y)$ نىڭ دەسلەپكى شەرت $y|_{x=x_0}=y_0$ ئاستىدىكى ئالاھېدە يېشىمىنى تېپىش مەسىلىسىنى كۆرسىتىدۇ. يەنى:
$$
\left\{\begin{array}{l}
y' = f(x,y) \\
y|_{x=x_0}=y_0
\end{array}\right.
$$

\item \textbf{ئىككىنچى دەرىجىلىك دەسلەپكى قىممەت مەسىلىسى}\\
تەڭلىمە $y^{''}=f(x,y,y')$ نىڭ دەسلەپكى شەرت 
$y|_{x=x_0}=y_0, y'|_{x=x_0}=y_{0}'$
 ئاستىدىكى ئالاھېدە يېشىمىنى تېپىش مەسىلىسىنى كۆرسىتىدۇ. يەنى:
$$
\left\{\begin{array}{l}
y^{''}=f(x,y,y') \\
y|_{x=x_0}=y_0, y'|_{x=x_0}=y_{0}'
\end{array}\right.
$$
\item \textbf{ئاجراتقىلى بولىدىغان تەڭلىمە}\\
شەكلى تۆۋەندىكىدەك بولغان تەڭلىمىنى ئاجراتقىلى بولىدىغان تەڭلىمە دەپ ئاتايمىز:
$$y'=f(x)g(y)$$
بۇنى يېشىشتە، ئوخشاش مىقدارلارنى بىىر تەرەپكە يىغىپ ئىنتېگىراللىساقلا بولىدۇ.
\begin{colorful}[green]
شەكلى
$\dfrac{\textrm{d}y}{\textrm{d}x}=f(ax+by+c)$
كە ئوخشاش تەڭلىمىدە،($a,b,c$لار بىرلا ۋاقىتتا نۆل ئەمەس).
$u=ax+by+c$
ئۇنداقتا
$\dfrac{\textrm{d}u}{\text{d}x}=a+b\dfrac{\textrm{d}y}{\textrm{d}x}$
بولىدۇ، بۇنى ئەسلىدىكى تەڭلىمىگە بېرىكتۈرگەندە 
$\dfrac{\textrm{d}u}{\textrm{d}x}=a+bf(u)$
گە ئېرىشىمىز، بۇ دەل ئاجراتقىلى بولىدىغان تەڭلىمە.
\end{colorful}
%%%%%%%%%%
\begin{myexample}
	تەڭلىمە
	$\dfrac{dy}{dx}=2xy$
	نى يېشىڭ.
	\\\rule{\linewidth}{0.05em}\\
	كۆرۈۋېلىشقا بولىدۇكى، بۇ بىر ئاجراتقىلى بولىدىغان تەڭلىمە.\\
	$$\displaystyle{\int\dfrac{dy}{y}}=\int2x\,dx, \ln\vert y\vert=x^2+C, \vert y\vert=e^{x^2+C}$$
	$$\therefore y=\pm e^{x^2}e^C=\pm C_1e^{x^2}=C_2e^{x^2}$$
\end{myexample}

\item \textbf{بىر جىنىسلىق تەڭلىمە}\\
شەكلى 
$y'=f(x,y)=\phi(\frac{y}{x})$
بولغان تەڭلىمە بىر جىنىسلىق تەڭلىمە دەپ ئاتىلىدۇ.\\
بۇنى يېشىشىنىڭ باسقۇچلىرى:
$$u=\dfrac{y}{x}, y=xu, \dfrac{\textrm{d}y}{\textrm{d}x}=u+x\dfrac{\textrm{d}u}{\textrm{d}x}$$
شۇنىڭ بىلەن:
$$
u+x\dfrac{\textrm{d}u}{\textrm{d}x}=\varphi(u),\quad \therefore x\dfrac{\textrm{d}u}{\textrm{d}x}=\varphi(u)-u
$$
بۇنى پارچىلىغىلى بولىدىغان تەڭلىمە يېشىش ئۇسۇلى بويىچە يېشىشكە بولىدۇ.

\begin{colorful}[pink]
شەكلى
$\dfrac{\textrm{d}y}{\textrm{d}x}=\dfrac{A_1x+B_1y}{A_2x+B_2y}$
بولغان تەڭلىمىنى، تەڭلىكنىڭ ئىككى تەرىپىگە
$x$
نى بۆلۈش ئارقىلىق بىر جىنىسلىق تەڭلىمىگە ئېرىشەلەيمىز.\\
شەكلى 
$\dfrac{\textrm{d}y}{\textrm{d}x}=\dfrac{A_1x+B_1y+C_1}{A_2x+B_2y+C_2}$
بولغان تەڭلىمىدە، ئاۋال تۇراقلىق سان
$C$
نى
$x=X+h,y=Y+k$
ئارقىلىق ئالماشتۇرۇش ئېلىپ بارغاندا،
$$\dfrac{\textrm{d}Y}{\textrm{d}X}=\dfrac{A_1X+B_1Y+A_1h+B_1k+C_1}{A_2X+B_2Y+A_2h+B_2k+C_2}$$
بۇنىڭدا مۇۋاپپىق سان
$h,k$
بىلەن 
$A_1h+B_1k+C_1=A_2h+B_2k+C_2=0$
نى قانائەتلەندۈرۈپ، بىر جىنىسلىق تەڭلىمىگە ئېرىشەلەيمىز. بۇ ۋاقىتتا 
$\dfrac{A_2}{A_1}\neq\dfrac{B_2}{B_1}$
بولغاندا تېپىپ چىقىشقا بولىدۇكى
$$\left\{\begin{array}{l}
	k=\dfrac{A_1C_2-A_2C_1}{A_2B_1-A_1B_2} \\  
	h=\dfrac{A_1B_1C_2-A_2B_1C_1+A_1A_2B_1C_1-A_1^2B_2C_1}{A_1^2B_2-A_1A_2B_1}
\end{array}
\right.$$
ئەگەر 
$\dfrac{A_2}{A_1}=\dfrac{B_2}{B_1}=\lambda$
بولغاندا، 
$A_1x+B_1y=v$
قىلىپ خاتېرلىۋالساق، كەلتۈرۈپ چىقىرىشقا بولىدۇكى
$$
\dfrac{\textrm{d}v}{\textrm{d}x}=A_1+B_1\,\dfrac{\textrm{d}y}{\textrm{d}x}
=\dfrac{\textrm{d}v}{\textrm{d}x}=A_1+B_1\dfrac{v+C_1}{\lambda v+C_2}
=\dfrac{(A_1\lambda+B_1)v+A_1C_2+B_1C_1}{\lambda v+C_2}
$$
بۇنى ئاجراتقىلى بولىدىغان تەڭلىمە بويىچە يېشىمىز.


\end{colorful}

%%%%%%%%%
\begin{myexample}
	تەڭلىمە
$y^2+x^2\dfrac{\textrm{d}y}{\textrm{d}x}=xy\dfrac{\textrm{d}y}{\textrm{d}x}$
	نى يېشىڭ.
	\\\rule{\linewidth}{0.05em}\\
	ئەزا يۆتكەش ئارقىلىق
	$\dfrac{\textrm{d}y}{\textrm{d}x}=\dfrac{y^2}{xy-x^2}$
	گە ئېرىشەلەيمىز.\\سول تەرەپ سۈرئەت مەخرەجنى 
	$x^2$
	غا بۆلۈش ئارقىلىق
	$\dfrac{\dfrac{y^2}{x^2}}{\dfrac{xy-x^2}{x^2}}=\dfrac{\left(\dfrac{y}{x}\right)^2}{\dfrac{y}{x}-1}$
	گە ئېرىشەلەيمىز.\\كۆرۈۋېلىشقا بولىدۇكى، بۇ $\dfrac{y}{x}$ شەكىلدىكى بىر جىنىسلىق تەڭلىمە. ئەگەر
	$u+x\dfrac{\textrm{d}u}{\textrm{d}x}=\dfrac{u^2}{u-1}$

	$$x\dfrac{\textrm{d}u}{\textrm{d}x}=\dfrac{u^2}{u-1}-u=\dfrac{u}{u-1},
	\therefore\dfrac{u-1}{u}\textrm{d}u=\dfrac{\textrm{d}x}{x}$$
	$$
	\therefore\displaystyle{\int\dfrac{u-1}{u}\textrm{d}u=\int\dfrac{\textrm{d}x}{x}}, u-\ln u=\ln x+C, \ln xu=u+C$$
	نەتىجىگە 
	$u=\dfrac{y}{x}$
	نى ئالماشتۇرساق
	$\ln y=\dfrac{y}{x}+C$
	،شۇڭا
	$y=Ce^{\frac{y}{x}}$
\end{myexample}

\end{itemize}


\section{ئادەتتىكى دېففېرېنسىئال تەڭلىمە}
ئادەتتىكى دىففېرېنسىيال تەڭلىمە (ODE) دېگىنى دىففېرېنسىيال تەڭلىمىدىكى نامەلۇم مىقدار يەككە ئەركىن ئۆزگەرگۈچى مىقدارنىڭ فۇنكسىيەسى ئىكەنلىكىنى كۆرسىتىدۇ. ئەڭ ئاددىي ئادەتتىكى دىففېرېنسىئال تەڭلىمە، نامەلۇم مىقدار بىر ھەقىقىي سان ياكى كومپلېكس ساننىڭ فۇنكسىيىسى بولۇشى مۇمكىن، لېكىن نامەلۇم مىقدار بىر ۋىكتور فۇنكسىيىسى ياكى ماترىتسا فۇنكسىيىسى بولۇشى مۇمكىن، كېيىنكىسى ئادەتتىكى دىففېرېنسىئال تەڭلىمىدىن تەركىب تاپقان تەڭلىمىلەر سىستېمىغا ماس كېلىدۇ. ئەڭ كۆپ ئۇچرايدىغان ئىككى خىلى بىرىنچى تەرتىپلىك دىففېرېنسىيال تەڭلىمە ۋە ئىككىنچى تەرتىپلىك دىففېرېنسىيال تەڭلىمىدىن ئىبارەت.
 
\subsection{سىزىقلىق دىففېرېنسىئال تەڭلىمە}
شەكلى 
$\dfrac{\textrm{d}y}{\textrm{d}x}+P(x)y=Q(x)$
بولغان تەڭلىمە بىرىنچى تەرتىپلىك سىزىقلىق دىففېرېنسىيال تەڭلىمە دېيىلدۇ. تەڭلىمىدىكى نامەلۇم فۇنكسيە $y$ ۋە ئۇنىڭ ھاسىلىسىنىڭ دەرىجىسى بىرىنچى دەرىجە.\\
ئەگەر $Q(x)=0$ بولغاندا، بۇ بىرىنچى تەرتىپلىك بىر جىنىسلىق دىففېرېنسىيال تەڭلىمە دېيىلىدۇ. بۇ ۋاقىتتا 
$$\dfrac{\textrm{d}y}{y}=-P(x)\,\textrm{d}x,\ln y=\int P(x)\,\textrm{d}x+C',y=e^{-\int P(x)\,\textrm{d}x}\cdot e^{C'},y=Ce^{-\int P(x)\,\textrm{d}x}$$
ئەگەر 
$Q(x)\neq 0$
بۇنى \textbf{تۇراقلىق ساننى ئالماشتۇرۇش ئۇسۇلى } ئارقىلىق يېشىشكە بولىدۇ. بۇنىڭ قەدەم باسقۇچلىرى تۆۋەندىكىچە:\\
ئەگەر 
$Q(x)=0$
بولغاندا، تەڭلىمە يېشىمى
$y=Ce^{-\int P(x)\,\textrm{d}x}$
بولاتتى، ئەمما $Q(x)\neq 0$
بولغاچقا، بۇيەردىكى تۇراقلىق سان $C$ دەل $x$ نىڭ فۇنكىسىيەسى بولىدۇ، شۇڭا 
$y=ue^{-\int P(x)\,\textrm{d}x}$
بولغاندا، بۇنى ئەسلى تەڭلىمىگە باغلاش ئارقىلىق
$$u'e^{-\int P(x)\,\textrm{d}x}-ue^{-\int P(x)\,\textrm{d}x}P(x)+P(x)ue^{-\int P(x)\,\textrm{d}x}=Q(x)$$
گە ئېرىشەلەيمىز. يەنى
$u'e^{-\int P(x)\,\textrm{d}x}=Q(x)$
بۇنى $u'$ گە نىسبەتەن ئىنتېگراللىساق
$$
u=\displaystyle{\int Q(x)e^{\int P(x)\,\textrm{d}x}\,\textrm{d}x}+C
$$
ئاخىرىدا بۇنى ئەسلى تەڭلىمىگە باغلاش ئارقىلىق فورمۇلا
\begin{center}
	\tcboxmath[colback=lightblue!25!white,colframe=blue]{y=e^{-\int P(x)\,\textrm{d}x}(\int Q(x)e^{\int P(x)\,\textrm{d}x}\,\textrm{d}x+C)}
\end{center}

گە ئېرىشەلەيمىز. دىمەك بىر جىنىسلىق بولمىغان تەڭلىمىنىڭ يېشىمى، بىر جىنىسلىق تەڭلىمىنىڭ ئورتاق يېشىمىگە بىر جىنىسلىق بولمىغان تەڭلىمىنىڭ خاس يېشىمىنى قوشقانغا باراۋەر.

%%%%%%%%%
\begin{myexample}
	تەڭلىمە
	$\dfrac{\textrm{d}y}{\textrm{d}x}=\dfrac{1}{x+y}$
	نى يېشىڭ.
	\\\rule{\linewidth}{0.05em}\\
	تەڭلىمىدە $y$ نى ئاجرىتىپ چىقارغىلى بولمايدۇ، چۈنكى بۇ ئاجراتقىلى بولىدىغان تەڭلىمە ئەمەس. بۇنى شەكىل ئۆزگەرتىش ئارقىلىق
	$\dfrac{\textrm{d}y}{\textrm{d}x}-y=x$
	گە ئېرىشەلەيمىز. بۇ دەل سىزىقلىق دىففېرېنسىئال تەڭلىمە.\\
	بۇنىڭدا $x+y=u$ قىلىپ خاتىرلىۋالساق،
$$y=u-x, \dfrac{\textrm{d}y}{\textrm{d}x}=\dfrac{\textrm{d}u}{\textrm{d}x}-1, \dfrac{\textrm{d}u}{\textrm{d}x}=\dfrac{1+u}{u}, \dfrac{u}{1+u}\textrm{d}u=\textrm{d}x$$
بۇنى ئاجراتقىلى بولىدىغان تەڭلىمە بويىچە بىرتەرەپ قىلىساق بولىدۇ.
\end{myexample}

\subsection{بېرنوئىل تەڭلىمىسى}
شەكلى
$\dfrac{\textrm{d}y}{\textrm{d}x}+P(x)y=Q(x)y^n$
بولغان تەڭلىمىنى بېرنوئىل تەڭلىمىسى دەپ ئاتايمىز. بۇنىڭدا، ئەگەر $y=0$ بولسا دەل بىر جىنىسلىق تەڭلىمە، $y=1$ بولسا ئاجراتقىلى بولىدىغان تەڭلىمە ھېساپلىنىدۇ.بېرنوئىل تەڭلىمىسىنى يېشىشنىڭ ئۇسۇلى تۆۋەندىكىچە:
\begin{colorful}[cyan]
ئالدى بىلەن شەكىل ئۆزگەرتىمىز، يەنى 
$y^{-n}\dfrac{\textrm{d}y}{\textrm{d}x}+P(x)y^{1-n}=Q(x)$\\
بۇنىڭدا
$y^{1-n}=z$
دەپ خاتېرلىۋالساق،
$\dfrac{\textrm{d}z}{\textrm{d}x}=(1-n)y^{-n}\dfrac{\textrm{d}y}{\textrm{d}x}$
بولىدۇ.\\
شۇڭا 
$\dfrac{1}{1-n}\dfrac{\textrm{d}z}{\textrm{d}x}=y^{-n}\dfrac{\textrm{d}y}{\textrm{d}x}$
گە ئېرىشەلەيمىز.\\
بۇنى ئەسلىدىكى تەڭلىمىگە باغلىساق: 
$$\dfrac{\textrm{d}y}{\textrm{d}x}+P(x)y=Q(x)y^n$$
بۇنىڭدىن كەلتۈرۈپ چىقىرىمىز:
$$\dfrac{1}{1-n}\dfrac{\textrm{d}z}{\textrm{d}x}+P(x)z=Q(x)$$,
$$\dfrac{\textrm{d}z}{\textrm{d}x}=(1-n)P(x)z=(1-n)Q(x)$$
دىمەك سىزىقلىق دىففېرېنسىئال تەڭلىمىگە ئايلاندى.
\end{colorful}

\begin{myexample}
	تەڭلىمە
	$y\,\textrm{d}x=(1+x\ln y)x\,\textrm{d}y$($y>0$)
	نى يېشىڭ.
	\\\rule{\linewidth}{0.05em}\\
	كەسىر شەكلىگە ئايلاندۇرساق: 
	$$\dfrac{\textrm{d}x}{\textrm{d}y}=\dfrac{(1+x\ln y)x}{y}=\dfrac{1}{y}x+\dfrac{\ln y}{y}x^2$$
	بېرنوئىل تەڭلىمىسىىدە:
	\begin{align*}
		x'+P(x)x&=Q(x)x^n, x'-\dfrac{1}{y}x=\dfrac{\ln y}{y}x^2\\
		P(x)&=-\dfrac{1}{y}, Q(x)=\dfrac{\ln y}{y}
	\end{align*}
تەڭلىكنىڭ ئىككى تەرىپىنى
	$x^{-2}$
	گە كۆپەيتسەك:
	$$x^{-2}x'-\dfrac{1}{y}x^{-1}=\dfrac{\ln y}{y}, \quad z=x^{-1}, \dfrac{\textrm{d}z}{\textrm{d}y}=-\dfrac{1}{x^2}\dfrac{\textrm{d}x}{\textrm{d}y}$$
	ئەسلى تەڭلىمىگە باغلىساق:
	$$-\dfrac{\textrm{d}z}{\textrm{d}y}-\dfrac{1}{y}z=\dfrac{\ln y}{y}, \quad \dfrac{\textrm{d}z}{\textrm{d}y}+\dfrac{1}{y}z=-\dfrac{\ln y}{y}$$
	فورمۇلا ئارقىلىق:
	\begin{align*}
	z &= e^{-\int\frac{1}{y}\textrm{d}y}\left(\displaystyle{\int e^{\int\frac{1}{y}\textrm{d}y}\cdot\left(\dfrac{\ln y}{y}\right)+C}\right)\\
	&=  \dfrac{1}{y}(-\int\ln y\,\textrm{d}y+C)\\
	&= \dfrac{1}{y}(-y(\ln y-1)+C)\\
	&= -\ln y+1+\dfrac{C}{y}\\
	&\therefore x = \dfrac{y}{-y\ln y+y+C}
	\end{align*}
\end{myexample}

\subsection{تۆۋەنلەتكىلى بولىدىغان دېففېرېنسىئال تەڭلىمە}
 ئادەتتە بەزى يۇقىرى دەرىجىلىك تەڭلىمىلەرنى تۆۋەن دەرىجىلىك تەڭلىمىلەرنى چۈشۈرۈپ يېشىشكە بولىدۇ ،  بىز بۇنداق دەرىجىسىنى تۆۋەنلەتكىلى بولىدىغان تەڭلىمىلەرنى تۆۋەنلەتكىلى بولىدىغان دېففېرېنسىئال تەڭلىمە دەيمىز.
\begin{itemize}
\item[\faIcon{hand-point-left}] $y^{(n)}=f(x)$\\
تەڭلىمىنىڭ ئوڭ تەرىپىدە پەقەت $x$ لا بار بولغان فۇنكىسىيە.\\
بۇ خىلدىكى تەڭلىمىدە، ئارقىمۇ-ئارا ھاسىلىسىنى ھىساپلىغاندا $n$ دانە تۇراقلىق مىقدارنى ئۆز ئىچىگە ئالغان ئورتاق يېشىمىگە ئېرىشەلەيمىز. تۆۋەندىكى مىسالدا بېرىلگەندەك:
\begin{myexample}
	تەڭلىمە $y'''=e^{2x}-\cos x$ نى يېشىڭ.
	\\\rule{\linewidth}{0.05em}
	\begin{align*}
	y''&=\dfrac{1}{2}e^{2x}-\sin x+C_1, y'=\dfrac{1}{4}e^{2x}+\cos x+C_1x+C_2\\
	y&=\dfrac{1}{8}e^{2x}+\sin x+\dfrac{1}{2}C_1x^2+C_2x+C_3
	\end{align*}
\end{myexample}

\item[\faIcon{hand-point-left}] $y''=f(x,y')$\\
بۇ خىلدىكى تەڭلىمىدە $y'',y',x$ لەر بار، ئەمما $y$ يوق. بۇنداق تەڭلىمىلەردە $y'=p$ قىلىپ خاتېرلىۋالساق، $y''=p'$ بولىدۇ، بۇنى ئەسلى تەڭلىمىگە باغلىساق، $p$ غا مۇناسىۋەتلىك بىرىنجى دەرىجىلىك تەڭلىمىگە ئېرىشەلەيمىز.

\begin{myexample}
	تەڭلىمە $(1+x^2)y''=2xy'$، دەسلەپكى شەرتى 
	$y\vert_{x=0}=1,y'\vert_{x=0}=3$
	\\\rule{\linewidth}{0.05em}
\begin{align*}
	y'&=p, y''=p', (1+x^2)p'=2xp\\
	\dfrac{\textrm{d}p}{p}&=\dfrac{2x}{1+x^2}\textrm{d}x\\
	\ln p&=\ln(1+x^2)+C', p=C(1+x^2)\\
	y'&=3(1+x^2), y=x^3+3x+1
\end{align*}

\end{myexample}

\item[\faIcon{hand-point-left}] $y''=f(y,y')$\\
بۇ خىلدىكى تەڭلىمىدە $y'',y',y$ لەر بار، ئەمما $x$ يوق.\\
بۇ خىلدىكى تەڭلىمىلەردە، ئالدىنقىسىغا ئوخشاش $y'=p$ قىلىپ خاتېرلىۋالساق، 
$$y''=p'=\dfrac{\textrm{d}p}{\textrm{d}x}=\dfrac{\textrm{d}p}{\textrm{d}y}\cdot\dfrac{\textrm{d}y}{\textrm{d}x}=p\dfrac{\textrm{d}p}{\textrm{d}y}=f(y,p)$$
بۇنى ئاجراتقىلى بولىدىغان تەڭلىمە بويىچە يېشىشكە بولىدۇ.

\end{itemize}


\section{يۇقرى دەرىجىلىك سىزىقلىق دىففېرېنسىئال تەڭلىمە}
\subsection{يۇقرى دەرىجىلىك تەڭلىمە}
بىرىنجى پاراگىرافتا ئادەتتىكى دېففېرىنسىئال تەڭلىمە خاتېرلەندى.\\
ئىككىنچى پاراگىرافتا تۆۋەنلەتكىلى بولىدىغان تەڭلىمىلەر خاتېرلەندى.\\
بۇ بۆلەكتە شەكلى
$y^{(n)}+a_1(x)y^{(n-1)}+\cdots+a_{n-1}(x)y'+a_n(x)y=0$
ۋە
$y^{(n)}+a_1(x)y^{(n-1)}+\cdots+a_{n-1}(x)y'+a_n(x)y=f(x)$
بولغان $n$ دەرجىلىك تەڭلىمە تونۇشتۇرىلىدۇ.

\subsection{ئەيلېر تەڭلىمىسى}

\begin{MyDefinition}{ئەيلېر تەڭلىمىسى}{}
شەكلى
$$x^2\dfrac{\textrm{d}^2y}{\textrm{d}x^2}+px\dfrac{\textrm{d}y}{\textrm{d}x}+qy=f(x)$$
بولغان تەڭلىمە ئەيلېر تەڭلىمىسى دەپ ئاتىلىدۇ.
\end{MyDefinition}
تەڭلىمىدە $p,q$ لار ئېنىق بولغان تۇراقلىق سان، $f(x)$ ئېنىق بولغان فۇنكىسىيە.
ئەيلېر تەڭلىمىسىنى يېشىشتە تۆۋەندىكىدەك ئىككى باسقۇچقا بۆلۈشكە بولىدۇ.\\
ئەگەر$x>0$ بولسا:\\
$x=e^t$
دەپ خاتېرلىۋالساق،
$t=\ln x$
بولىدۇ.
\begin{align*}
\dfrac{\textrm{d}t}{\textrm{d}x}
&=\dfrac{1}{x},\dfrac{\textrm{d}y}{\textrm{d}x}=\dfrac{\textrm{d}y}{\textrm{d}t}\dfrac{\textrm{d}t}{\textrm{d}x}=\dfrac{1}{x}\dfrac{\textrm{d}y}{\textrm{d}t}
\\
\dfrac{\textrm{d}^2y}{\textrm{d}x^2}
&=\dfrac{\textrm{d}}{\textrm{d}x}\left(\dfrac{1}{x}\dfrac{\textrm{d}y}{\textrm{d}t}\right)
=-\dfrac{1}{x^2}\dfrac{\textrm{d}y}{\textrm{d}t}+\dfrac{1}{x}\dfrac{\textrm{d}}{\textrm{d}x}\left(\dfrac{\textrm{d}y}{\textrm{d}t}\right)\\
&=-\dfrac{1}{x^2}\dfrac{\textrm{d}y}{\textrm{d}t}+\dfrac{1}{x^2}\dfrac{\textrm{d}^2y}{\textrm{d}t^2}
\end{align*}
تەڭلىمە
$\dfrac{\textrm{d}^2y}{\textrm{d}t^2}+(p-1)\dfrac{\textrm{d}y}{\textrm{d}t}+qy=f(e^t)$
گە ئايلىنىدۇ، ئاخىردا
$t=\ln x$
بىلەن قايتۇرساق تەڭلىمە يېشىمىگە ئېرىشەلەيمىز.\\

ئەگەر 
$x<0$
بولسا،
$x=-e^t$
قىلىپ خاتېرلىۋالساق، ئۈستىدىكى ئۇسۇل بىلەن يېشىمىگە ئېرىشەلەيمىز.
\begin{myexample}
	تەڭلىمە
	$x^2\dfrac{\textrm{d}^2y}{\textrm{d}x^2}+4x\dfrac{\textrm{d}y}{\textrm{d}x}+2y=0, \quad x>0$
	نى يېشىڭ.
	\\\rule{\linewidth}{0.05em}\\
	بىۋاستە فورمۇلادىن پايدىلانساق،
	$\dfrac{\textrm{d}^2y}{\textrm{d}t^2}+3\dfrac{\textrm{d}y}{\textrm{d}t}+2y=0$\\
	شۇڭا
	$y''+3y'+2y=0$ \\
 خاراكتېرلىگۈچى تەڭلىمىسى:
	$$\lambda^2+3\lambda+2=0, \lambda_1=-1, \lambda_2=-2$$
	$$\therefore y=C_1e^{-x}+C_2e^{-2x}$$
	ئاخېرىدا
	$x=e^t $
	بىلەن ئالماشتۇرساق:
	$$y=\dfrac{C_1}{x}+\dfrac{C_2}{x^2}$$
	
\end{myexample}


