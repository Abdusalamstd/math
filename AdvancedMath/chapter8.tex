\mychapter{دىففېرېنسىئال تەڭلىمە}
\par\bigskip
\begin{tcolorbox}
تەڭلىمە ئۇقۇمى باشلانغۇچ ماتېماتىكىسىدا ئەڭ بۇرۇن ئۇچرايتتى. ئىلگىرىكى مەزمۇنلاردا فۇنكسىيە، ھاسىلە ئۇقۇمى، دىففېرېنسىئال ۋە ئىنتېگرال ئۇقۇملىرىنى ئىگەللىگەندىن كىيىن مۇشۇلارنىڭمۇ تەڭلىمىگە ئائىت قوللىنىشلىرىنى بىلىش ئۈچۈن، شۇنداقلا تۇرمۇشتىكى ئەمەلىي مەسىلىلەرنىڭ ئېھتىياجى ئۈچۈن تۆۋەندە يېڭى بىر بىلىم نۇقتىسى بىلەن تونۇشۇپ چىقىمىز. بۇ باپتا بىر قەدەر قىيىن بولغان نۇقتا
\textbf{دىففېرېنسىئال تەڭلىمە}
ھەققىدە دەسلەپكى بىلىملەرنى ئۆگىنىپ چىقايلى.

\end{tcolorbox}
\section{دىففېرېنسىئال تەڭلىمە}
\subsection{ئاساسىي ئۇقۇم}

\begin{MyDefinition}{دىففېرېنسىئال تەڭلىمە}{}
	نامەلۇم فۇنكىسىيە ۋە نامەلۇم فۇنكىسىيە ھاسىلىسىنىڭ ئۆزگەرگۈچى مىقدار ئارىسىدىكى مۇناسىۋىتىنى ئىپادىلەيدىغان تەڭلىمە. يەنى فۇنكىسىيە ھاسىلىسىنى ئۆز ئىچىگە ئالغان تەڭلىمە دىففېرېنسىئال تەڭلىمە دەپ ئاتىلىدۇ. بۇنى 
	$$F(x,y,y^{'},...,y^{(n)})=0$$
	ئارقىلىق خاتېرلەشكە بولىدۇ.
\end{MyDefinition}
دىففېرېنسىئال تەڭلىمىدىكى نامەلۇم فۇنكىسىيە ھاسىلىسىنىڭ دەرىجىسى، دىففېرېنسىئال تەڭلىمىنىڭ \textbf{دەرىجىسى} دەپ ئاتىلىدۇ.

\subsubsection{دىففېرېنسىئال تەڭلىمىنىڭ يېشىمى}
ئەگەر فۇنكىسىيە $y=\phi(x)$ نىڭ $n$ دەرىجىلىك ئۈزلۈكسىز ھاسىلىسى $\phi^n(x)$ ، بېرىلگەن ئىنتېرۋال $I$ دا مەۋجۇت ھەم تەڭلىمە
$$F(x,\phi(x),\phi^{'}(x),...,\phi^{(n)}(x))=0$$
نى قانائەتلەندۈرسە، ئۇنداقتا فۇنكىسىيە $y=\phi(x)$ تەڭلىمە
$$F(x,\phi(x),\phi^{'}(x),...,\phi^{(n)}(x))=0$$
نىڭ ئىنتېرۋال $I$ دىكى \textbf{يېشىمى} دەپ ئاتىلىدۇ. 
\subsubsection{ئومۇمىي يېشىمى}
ئەگەر دىففېرېنسىئال تەڭلىمە يېشىمى خالىغان تۇراقلىق ساننى ئۆز ئىچىگە ئالغان ھەمدە خالىغان تۇراقلىق ساننىڭ سانى تەڭلىمە دەرىجىسى بىلەن تەڭ بولغاندا، بۇ يېشىمىنى تەڭلىمىنىڭ \textbf{ئومۇمىي يېشىمى} دەپ ئاتايمىز.
\subsubsection{ئالاھېدە يېشىمى}
دىففېرېنسىئال تەڭلىمە ئومۇمىي يېشىمىدىكى خالىغان تۇراقلىق ساننى مۇقىم بېكىتكەندىن كىيىن ئېرىشكەن يېشىمنى، \textbf{ئالاھىدە يېشىمى} دەپ ئاتايمىز.

\subsection{ئاساسىي تەڭلىمىلەر}
\begin{itemize}
\item \textbf{دەسلەپكى قىممەت شەرتى}\\
ئەگەر $x=x_0$ بولغاندىكى فۇنكىسىيە ۋە ئۇنىڭ ھاسىلىسىنىڭ قىممىتى $y_0,y_{0}^{'}$ بېرىلگەن بولسا، بۇنداق شەرتلەرنى بىز تەڭلىمىنىڭ دەسلەپكى قىممەت شەرتى دەپ ئاتايمىز.
\item \textbf{بىرىنچى دەرىجىلىك دەسلەپكى قىممەت مەسىلىسى}\\
تەڭلىمە $y^{'}=f(x,y)$ نىڭ دەسلەپكى شەرت $y|_{x=x_0}=y_0$ ئاستىدىكى ئالاھېدە يېشىمىنى تېپىش مەسىلىسىنى كۆرسىتىدۇ. يەنى:
$$
\left\{\begin{array}{l}
y^{'} = f(x,y) \\
y|_{x=x_0}=y_0
\end{array}\right.
$$

\item \textbf{ئىككىنچى دەرىجىلىك دەسلەپكى قىممەت مەسىلىسى}\\
تەڭلىمە $y^{''}=f(x,y,y^{'})$ نىڭ دەسلەپكى شەرت $y|_{x=x_0}=y_0, y^{'}|_{x=x_0}=y_{0}^{'}$ ئاستىدىكى ئالاھېدە يېشىمىنى تېپىش مەسىلىسىنى كۆرسىتىدۇ. يەنى:
$$
\left\{\begin{array}{l}
y^{''}=f(x,y,y^{'}) \\
y|_{x=x_0}=y_0, y^{'}|_{x=x_0}=y_{0}^{'}
\end{array}\right.
$$
\item \textbf{پارچىلىغىلى بولىدىغان تەڭلىمە}\\
شەكلى تۆۋەندىكىدەك بولغان تەڭلىمىنى پارچىلىغىلى بولىدىغان تەڭلىمە دەپ ئاتايمىز:
$$
y^{'}=f(x)g(y)
$$
\end{itemize}

سالام دۇنيا.
%%%%%%%%%%
\begin{myexample}
	تەڭلىمە
	$\dfrac{dy}{dx}=2xy$
	نى يېشىڭ.
	\\\rule{\linewidth}{0.05em}\\
	كۆرۈۋېلىشقا بولىدۇكى، بۇ بىر پارچىلىغىلى بولىدىغان تەڭلىمە.\\
$$\displaystyle{\int\dfrac{dy}{y}}=\int2x\,dx, \ln\vert y\vert=x^2+C, \vert y\vert=e^{x^2+C}$$
$$\therefore y=\pm e^{x^2}e^C=\pm C_1e^{x^2}=C_2e^{x^2}$$
\end{myexample}



\section{ئادەتتىكى دېففېرېنسىئال تەڭلىمە}
\subsection{سىزىقلىق دىففېرېنسىئال تەڭلىمە}
\subsection{بېرنوئىل تەڭلىمىسى}
\subsection{تۆۋەنلەتكىلى دېففېرېنسىئال تەڭلىمە}

\section{يۇقرى دەرىجىلىك سىزىقلىق دىففېرېنسىئال تەڭلىمە}
\subsection{يۇقرى دەرىجىلىك تەڭلىمە}
\subsection{ئەيلېر تەڭلىمىسى}


