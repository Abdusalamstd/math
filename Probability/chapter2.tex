\mychapter{دەسلەپكى بايان}

\par\bigskip
\begin{tcolorbox}
ھەممىمىزگە مەلۇم، ئۇچۇر تېخنىكىسى ئۇچقاندەك تەرەققىي قىلىۋاتقان بۇ دەۋردە كومپيۇتېردىن ئىبارەت بۇ ۋاقىت بۆلگۈچ ئەھمىيەتكە ئىگە كەشپىياتتىن ئۈنۈملۈك پايدىلىنىش ھەر ساھە ھەر كەسىپتىكىلەرنىڭ كەم بولسا بولمايدىغان تەلىپىگە ئايلاندى.
بۇ باپتا بىز \PY تىلى ھەققىدە ئاساسىي چۈشەنچە ۋە بىلىملەرنى ئۆگىنىپ چىقىمىز، بىرلىكتە \PY تىلى يېزىش مۇھېتى تەييارلاش، ئاددى \PY كودى ھەم ئۈچىنچى تەرەپ بولاق قاچىلاش توغرىسىدا تەپسىلىي بىلىملەرنى ئۆگىنىپ چىقىمىز.

\end{tcolorbox}

\section{تىل ھەققىدە}

بىز ئۈگەنمەكچى بولغان تىل  pythonبولۇپ ،س تىل قاتارلىق تىللاردىن ئالاھېدە پەرىقلىنىدۇ، شۇنداقلا بىر قەدەر ئاددى بولغان ھەممىلا ئادەم ئاسان ئۈگىنىۋالالايدىغان تىل.بۇ تىل 1991 يىلى ئىلان قىلىنغان بولۇپ تاھازېرغىچە كەڭ قوللۇنۇلۇپ كەلمەكتە. 2000يىلى 2.0 نەشىرى تارقىتىلدى ، بۇنىڭ ئىچىدە 2.7 نەشىرى مۇقىم نەشىرى دەپ قارىلىدۇ. 2008 يىلىدا 3.نەشىرى تارقىتىلدى ، بۇنىڭ ئىچىدە بىز 3.7 نەشىرىنى ئۆگىنىمىز ،ھازىرقى ئەڭ يېڭى نەشىرى بولسا 3.8.1.بىز نىمىشقا 3.7 نى تاللاپ ئۆگىنىمىز دىگەندە(قۇرۇق گەپ قىلماي لوخ گەپنى قىلسام،مەن كومپيۇتېرىمغا مۇشۇ نەشىرىنى قاچىلىغان) بۇ بىر 3 ئەۋلاد نەشىرىنىڭ بىر تۈرى، بۇنىڭ گىرامماتىكىسى 3 ئەۋلاد باشقا نەشىرىنىڭ گىرامماتىكىسى بىلەن ئوخشاش ،ھەم بىر قەدەر تونۇش .بۇ نەشىرى بىرقەدەر پىشقان نەشىرى ھېساپلىنىدۇ . 

\begin{multicols}{2}
قىسقا چۈشەنچە:\\
پىروگىرامما تىلى: \\Python
ئوقۇلۇشى: [ˈpaɪθɑːn]، پايسون،پايسېن،پەيسېن…\\
مەنىسى: بوغما يىلان\\
يېشى: 1991 دۇنياغا كۆز ئاچقان\\
تۈرى: ئالىي دەرىجىلىك كومپۇتېر پىروگىرامما تىلى\\
سىنبەلگىسى:\\
\begin{center}
\includegraphics[width=150pt, height=100pt]{chapter/python.eps}
\end{center}
\end{multicols}

\section{قىسقىچە تەرەققىيات تارىخى}
Python 1980-يىللارنىڭ ئاخىرى ۋە 90-يىللارنىڭ بېشىدا گوللاندىيە دۆلەتلىك ماتېماتىكا ۋە كومپيۇتېر ئىلمى ئىنستىتۇتىدا گۇيدو ۋان روسۇم تەرىپىدىن لايىھەلەنگەن.
Python نىڭ ئۆزى يەنە ABC ، Modula-3 ، C ، C ++ ، Algol-68 ، SmallTalk ، Unix قېپى ۋە باشقا ئورگىنال تىللىرى قاتارلىق نۇرغۇن تىللار تەرىپىدىن ئىجاد قىلىنغان.
Perl تىلىغا ئوخشاش ، Python مەنبە كودى يەنە GPL (GNU ئادەتتىكى ئاممىۋى ئىجازەتنامە) كېلىشىمىگە ئەمەل قىلىدۇ.\\
Python ھازىر يادرولۇق تەرەققىيات ئاچقۇچىلار گۇرۇپپىسى تەرىپىدىن قوغدىلىدۇ ، گۇيدو ۋان روسۇم يەنىلا ئۇنىڭ تەرەققىياتىغا يېتەكچىلىك قىلىشتا ئىنتايىن مۇھىم رول ئوينايدۇ.
Python 2.0  2000-يىلى 10-ئاينىڭ 16-كۈنى تارقىتىلدى ، بۇ ئەخلەتلەرنى تولۇق يىغىش ئىقتىدارى ئەمەلگە ئاشۇرۇلدى ، يۇنىكودنى قوللايدۇ.\\
Python 3.0  2008-يىلى 12-ئاينىڭ 3-كۈنى ئېلان قىلىنغان. بۇ نەشرى ئىلگىرىكى Python مەنبە كودىغا پۈتۈنلەي ماس كەلمەيدۇ. لىكىن قانداقلا بولمىسۇن ، نۇرغۇن يېڭى ئىقتىدارلار كېيىن كونا Python 2.6 / 2.7 نەشرىگە يۆتكەلدى، كونا پىروگىراممېرلانىڭ ئۇسلۇبلىرىنى ساقلاپ قېلىش ئەمەلگە ئاشۇرۇلدى.Python 3.0 نەشىرى كۆپىنچە Python 3000 ياكى قىسقارتىپ Py3k دەپ ئاتىلىدۇ. Python نىڭ ئىلگىرىكى نەشرىگە سېلىشتۇرغاندا ، بۇ بىر چوڭ يېڭىلاش.\\
Python 2.7 ئەڭ ئاخىرقى Python 2.x نەشرى دەپ بېكىتىلدى ، Python 2.x گرامماتىكىسىنى قوللىغاندىن باشقا ، ئۇ يەنە Python 3.1 گرامماتىكىسىنىڭ بىر قىسمىنى قوللايدۇ.\\

\subsection{بۇ تىلنىڭ ئالاھېدىلىكى}
1.تىلنى ئىگەللەش ئىنتايىن ئاسان،گىرامماتىكىسىمۇ ئاسان،رەسىم سىزغىلىمۇ بولىدۇ.\\
2.ماسلىنىشچانلىقى كۈچلۈك ،ئۇ سۇپىدىن بۇ سۇپىغا يۆتكەپ ئىشلىتىشكە بولىدۇ.\\
3.لايھەلەش ئاسان،ھالقىلىق سۆزلەر ئاسان،ئېسىدە تۇتىۋىلىش قولايلىق.\\
4.ئىزاھات خاراكتېردىكى تىل،يەنى بىر پىروگراممىنىڭ توغرا بولغان يېرى ئۈنۈملۈك يۈرىدۇ.\\
5.ئىنتېرئاكتىپ(ئۆز-ئارا)ھالەتتىمۇ يۈرەلەيدۇ،پاراڭلاشقاندەكلا پىروگىرامما يېزىشقا بولىدۇ.\\
6.شەيئىي ئوبىكتىپقا يۈزلەنگەن تىل،تەبئەتتىكى بارلىق شەيئىي ئۇنىڭ ئوبىكتى بولالايدۇ.\\
7.س تىلغا قارىغاندا يېڭى ئۈگەنگۈچىلەر ئاسان ئۆگىنەلەيدۇ،ئاجىز سان-سىپىرلىق تىل.\\
8.يەنە باشقا باشقا ئالاھېدىلىكلىرىمۇ بار.\\
بۇ تىلنىڭ ئالاھىدىلىكى يۇقىرىقىلاردىن تېخىمۇ جىق،ئالاھىلىكلىرى بۇنىڭ بىلەن تۈگىدى دېسەم  چوقۇم پىروگىراممېرلار كىتاپنى بىشىمغا ئاتىدۇ. ئېھتىمال ئۆزەممۇ ئۆزۈمنىڭ بېشىغا ئېتىپ قالارمەن.\\
\par
ئاددىيلىق - Python ئاددىيلىق ئىدىيىسىگە ۋەكىللىك قىلىدىغان تىل. ياخشى Python پروگراممىسىنى ئوقۇش گەرچە ئىنگلىزچە تەلەپلەر ئىنتايىن قاتتىق بولسىمۇ ، ئىنگلىزچە ئوقۇغاندەك ھېس قىلىدۇ! Python نىڭ ساختا كود خاراكتېرى ئۇنىڭ ئەڭ چوڭ ئەۋزەللىكىنىڭ بىرى. ئۇ سىزنى تىلنىڭ ئۆزىنى چۈشىنىشنىڭ ئورنىغا مەسىلىلەرنى ھەل قىلىشقا ئەھمىيەت بېرىدۇ.
\par
 ئۆگىنىش ئاسان - كۆرگىنىڭىزدەك ، Python نى ئۆگىنىش ناھايىتى ئاسان. يۇقىرىدا دېيىلگەندەك ، Python نىڭ ئىنتايىن ئاددىي گرامماتىكىسى بار.
\par
 ھەقسىز ۋە ئوچۇق مەنبە - Python FLOSS (ھەقسىز / ئوچۇق كود يۇمشاق دېتالى) نىڭ بىرى. ئاددىي قىلىپ ئېيتقاندا ، سىز بۇ يۇمشاق دېتالنىڭ كۆپەيتىلگەن نۇسخىسىنى ھەقسىز تارقىتالايسىز ، ئۇنىڭ ئەسلى كودىنى ئوقۇيالايسىز ، ئۇنىڭغا ئۆزگەرتىش ئېلىپ بارالايسىز ۋە ئۇنىڭ بىر قىسمىنى يېڭى ھەقسىز يۇمشاق دېتاللاردا ئىشلىتەلەيسىز. FLOSS گۇرۇپپا بىلىملىرىنى ئورتاقلىشىش ئۇقۇمىنى ئاساس قىلغان. بۇ Python نىڭ شۇنچە ياخشى بولۇشىدىكى سەۋەبلەرنىڭ بىرى ، ئۇ تېخىمۇ ياخشى Python نى كۆرۈشنى خالايدىغان بىر تۈركۈم كىشىلەر تەرىپىدىن ئىجاد قىلىنغان ۋە توختىماي ياخشىلانغان.
\par
 يۇقىرى سەۋىيىلىك تىل - Python دا پروگرامما يازغاندا ، پروگراممىڭىزدا ئىشلىتىلگەن ئىچكى ساقلىغۇچنى قانداق باشقۇرۇش قاتارلىق تۆۋەن دەرىجىلىك تەپسىلاتلارنى ئويلىشىڭىزنىڭ ھاجىتى يوق. ئېلىپ يۈرۈشكە ئەپلىك - ئوچۇق مەنبەلىك بولغانلىقى ئۈچۈن ، Python نۇرغۇن سۇپىلارغا يۆتكىلەلەيدۇ (ئۇ ئوخشىمىغان سۇپىلاردا ئىشلەش ئۈچۈن). ئەگەر سىستېمىغا تايىنىدىغان ئىقتىدارلارنى ئىشلىتىشتىن ساقلىنىشقا دىققەت قىلسىڭىز ، بارلىق Python پروگراممىلىرىڭىز تۆۋەندىكى سۇپىلارنىڭ خالىغان بىرىنى ئۆزگەرتمەي ئىجرا قىلالايدۇ. بۇ سۇپىلار Linux ، Windows ، FreeBSD ، Macintosh ، Solaris ، OS / 2 ، Amiga ، AROS ، AS / 400 ، BeOS ، OS / 390 ، z / OS ، Palm OS ، QNX ، VMS ، Psion ، Acom RISC OS ، VxWorks ، Linux نى ئاساس قىلغان PlayStation ، Sharp Zaurus ، Windows CE ھەتتا PocketPC ، Symbian ۋە گۇگۇلنىڭ ئاندىرويىد سۇپىسى!
\par
 ئىزاھات خاراكتېر - بۇ نۇقتا بەزى چۈشەندۈرۈشلەرگە موھتاج. C ياكى C ++ قاتارلىق تۈزۈلگەن تىلدا يېزىلغان پروگراممىنى ئەسلى ھۆججەتتىن (يەنى C ياكى C ++ تىلى) كومپيۇتېرىڭىز ئىشلىتىدىغان تىلغا (ئىككىلىك كود ، يەنى 0 ۋە 1) ئايلاندۇرغىلى بولىدۇ. بۇ جەريان تۈزگۈچى (تەرجىمىتېرمىكى) ۋە ئوخشىمىغان بەلگە ۋە تاللاشلار تەرىپىدىن تاماملىنىدۇ. پروگراممىڭىزنى ئىجرا قىلسىڭىز ، ئۇلىغۇچ / قايتا يۈكلەش يۇمشاق دېتاللىرى پروگراممىڭىزنى قاتتىق دىسكىدىن ئىچكى ساقلىغۇچقا كۆچۈرۈپ ئىجرا قىلىدۇ. Python تىلىدا يېزىلغان پروگراممىلارنى ئىككىلىك كودقا تۈزۈشنىڭ ھاجىتى يوق. پروگراممىنى ئەسلى كودتىن بىۋاسىتە ئىجرا قىلالايسىز. كومپيۇتېرنىڭ ئىچىدە Python تەرجىمانى ئەسلى كودنى بايتكود دەپ ئاتىلىدىغان ئارىلىق شەكىلگە ئايلاندۇرىدۇ ، ئاندىن ئۇنى كومپيۇتېر ئىشلىتىدىغان ماشىنا تىلىغا تەرجىمە قىلىپ ئىجرا قىلىدۇ. ئەمەلىيەتتە ، سىز ئەمدى پروگراممىلارنى قانداق تۈزۈش ، توغرا كۈتۈپخانىلارنىڭ ئۇلىنىشى ۋە كۆپەيتىلىشىگە قانداق كاپالەتلىك قىلىشتىن ئەنسىرىمىسىڭىزمۇ بولىدۇ ، بۇلارنىڭ ھەممىسى Python نى ئىشلىتىشنى ئاسانلاشتۇرىدۇ. سىز پەقەت Python پروگراممىڭىزنى باشقا كومپيۇتېرغا كۆچۈرۈشىڭىز كېرەك بولغاچقا ، ئۇ ئىشلەيدۇ ، بۇمۇ Python پروگراممىڭىزنى تېخىمۇ ئېلىپ يۈرۈشكە ئەپلىك قىلىدۇ.
\par
 ئوبيېكتقا يۈزلەنگەن - Python ھەم جەريانغا يۈزلەنگەن پروگرامما تۈزۈشنى ۋە ئوبيېكتقا يۈزلەنگەن پروگرامما تۈزۈشنى قوللايدۇ. « جەريانغا يۈزلەنگەن» تىللاردا ، پروگراممىلار پەقەت قايتا ئىشلىتىشكە بولىدىغان پروگرامما ياكى ئىقتىدارلاردىن ياسالغان. «ئوبيېكتقا يۈزلەنگەن» تىلدا پروگرامما سانلىق مەلۇمات ۋە ئىقتىداردىن تەركىب تاپقان ئوبيېكتپتىن ياسالغان. C ++ ۋە Java قاتارلىق باشقا ئاساسلىق تىللارغا سېلىشتۇرغاندا ، Python ناھايىتى كۈچلۈك ۋە ئاددىي ئۇسۇلدا ئوبيېكتقا يۈزلەنگەن پروگرامما تۈزۈشنى قوللايدۇ.
\par
 كېڭەيتىلىشچانلىقى - ئەگەر تېزرەك ئىجرا قىلىش ئۈچۈن ئاچقۇچلۇق كودىڭىزنىڭ بىر قىسمىغا ئېھتىياجلىق بولسىڭىز ياكى بەزى ئالگورىزىملارنىڭ ئاشكارا بولماسلىقىنى ئويلىسىڭىز ، پروگراممىڭىزنىڭ بىر قىسمىنى C ياكى C ++ دە يازالايسىز ، ئاندىن Python پروگراممىڭىزدا ئىشلىتەلەيسىز.
\par
 مول بولاقلار - Python ئۆلچەملىك بولاقلىرى ھەقىقەتەن چوڭ. ئۇ دائىملىق ئىپادىلەش ، ھۆججەت ھاسىل قىلىش ، بىرلىك سىنىقى ، تېما ، ساندان ، توركۆرگۈچ ، CGI ، FTP ، ئېلېكترونلۇق خەت ، XML ، XML-RPC ، HTML ، WAV ھۆججىتى ، پارول سىستېمىسى ، GUI قاتارلىق ھەر خىل ۋەزىپىلەرنى بىر تەرەپ قىلىشىڭىزغا ياردەم بېرەلەيدۇ. (گرافىكلىق ئىشلەتكۈچى كۆرۈنمە يۈزى) ، Tk ۋە باشقا سىستېمىغا مۇناسىۋەتلىك مەشغۇلاتلار. ئېسىڭىزدە تۇتۇڭ ، Python نى قاچىلىسىلا ، بۇ ئىقتىدارلارنىڭ ھەممىسىنى ئىشلەتكىلى بولىدۇ. بۇ Python نىڭ «تولۇق ئىقتىدارلىق» پەلسەپىسى دەپ ئاتىلىدۇ. ئۆلچەملىك بولاقلاردىن باشقا ، يەنە نۇرغۇنلىغان يۇقىرى سۈپەتلىك ئۈچىنچى تەرەپ بولاقلىرى بار.


\par
كەمچىللىكى:\\
1. Python كۆچمە قوللىنىشچان(Mobile Application) پروگرامما (مەسىلەن ئاندىرويد ، iOS ) ئېچىش ئۈچۈن ئانچە ياخشى ئەمەس.\\
2. Python پروگراممېرلىرى باشقا تىللارنى ئىشلىتىشتە قىيىنچىلىققا دۇچ كېلىدۇ.ئۇنىڭدىن سىرت تۇنجى بولۇپ Python تىلىنى ئۆگەنگەندىن كىيىن ،Python پىروگىراممېرچىلار باشقا تىلنى ئۆگىنىش خۇش ياقماس بوپ قاپتۇدەك.\\
3. Python نىڭ ئىچكى ساقلىغۇچ سەرپىياتى يۇقىرى.\\
4.Python is not used commonly in the Enterprise Development Sector\\

\section{قاچىلاش}
ئورگان تورى : https://www.python.org \\
ئاۋال ئورگان تورىغا چىقىپ چۈشۈرۈش بىتىدىن 3-نەشىرىنىڭ خالىغان بىر تۈرىنى چۈشۈرىمىز ،بۇ يەردە 3.7.1 نۇسخىسىنى چۈشۈرمىز. دىققەت:ۋىندوۋس7 ۋە ئىشخانلاردىكى ۋىدوۋس7 قاتارلىق 32 بىتلىق كومپيۇتېردا 32بىتلىق مۇۋاپپىق نەشىرىنى ،ۋىندوۋس10 ياكى 64 بىتلىق كومپيۇتېرلاردا 64 بىتلىق نەشىرىنى چۈشۈرىمىز.\\
چۈشۈرۈنگەندىن كىيىن قاچىلايمىز.قاچىلىغاندا بىز "كىيىنكى قەدەم"دىگەننى توختىماي باسساقلا كومپيۇتېر ئۆزى قاچىلايدۇ. قاچىلىغاندا ئاستىدىكى رەسىمدىكىدەك Add Python3.7 to PATH گە بەلگە تاللاپ قاچىلايمىز. بۇنى ئامال بار ئۇنتۇپ قالماسلىق كىرەك.

\par
قاچىلاپ بولۇپ، كومىيۇتېرنىڭ تېرمىنالىنى ئېچىپ تۆۋەندىكىدەك \PY نى يېزىپ سىناق قىلىمىز. مەسىلەن ۋىندوۋسىتا:\\
\begin{english}
\begin{windows}
Python`\\`Python 3.7.1 (v3.7.1:260ec2c36a, Oct 20 2018, 14:57:15) [MSC v.1915 64 bit (AMD64)] on win32`\\`Type "help", "copyright", "credits" or "license" for more information.`\\`>>>
\end{windows}
\end{english}
يۇقارقىدەك نەتىجە چىقىدۇ. Mac تە:\\
\begin{english}
\begin{macos}
 
Python
\end{macos}
\end{english}

بىرىنچى كود: \textenglish{Helo World!} \\
\begin{english}

\begin{windows}
Python`\\`Python 3.7.1 (v3.7.1:260ec2c36a, Oct 20 2018, 14:57:15) [MSC v.1915 64 bit (AMD64)] on win32`\\`Type "help", "copyright", "credits" or "license" for more information.`\\`>>> print("Hello World!")`\\`Hello World!`\\`>>> |
\end{windows}
\end{english}











